\usepackage[utf8]{inputenc}
\usepackage[T1]{fontenc}
\usepackage{helvet}% police helvetica
\usepackage{graphicx}%pour inserer des graphiques
	\graphicspath{{logo/}{input/}{mag62/articles/}} 
\usepackage[left=2.4cm,right=2.4cm,top=2cm,bottom=3cm]{geometry}%marges
\usepackage{lipsum}
\usepackage{array}
\usepackage{titlesec}
\usepackage{titletoc}
\usepackage{enumitem}
\usepackage{media9}
\usepackage{sistyle}
\addmediapath{mag62/articles/images}
\usepackage[switch, pagewise]{lineno}
\usepackage{fixltx2e}
\usepackage{nccrules}
\usepackage[bottom]{footmisc}
\usepackage{textgreek} % pour les lettres grecques, comme le symbol Omega
\usepackage{longtable} % pour les tableaux sur plusieurs page
\usepackage{wrapfig} % pour les logo dans les licences
\usepackage{setspace} % pour changer l'espace entre les lignes. Utilisé pour le pied de page

%\linenumbers


\usepackage{multirow}
\usepackage{amsmath}
\usepackage{amssymb,amsfonts}
\usepackage[warn]{textcomp}
\usepackage{color}
\usepackage{array}
\usepackage{supertabular}
\usepackage{hhline}
\makeatletter
\newcommand\arraybslash{\let\\\@arraycr}
\makeatother
\newcommand\liststyleLi{%
\renewcommand\labelitemi{•}
\renewcommand\labelitemii{◦}
\renewcommand\labelitemiii{${\blacksquare}$}
\renewcommand\labelitemiv{•}
}


%%%%%%%%entêtes et pieds de pages%%%%%%%%%%%%%%%%%%
\usepackage{fancyhdr}
\pagestyle{fancy}
\setlength{\headheight}{22.58pt}
\renewcommand{\headrulewidth}{0pt}


\makeatletter
\renewcommand{\footrulewidth}{0pt}
%%%%%%%%%%%%%%%%%%%%%%%%%%%%%%%%%%%%%%%%%%%%%%%%%%%%%%%%%%%%%%%%%%
%%%%%%%%%%%%%%%%%%%sections%%%%%%%%%%%%%%%%%%%%%%%%%%%%%%%
%\renewcommand{\thesubsection}{\@arabic\c@subsection}
%\renewcommand{\thesubsubsection}{\@arabic\c@subsection.\@arabic\c@subsubsection}
\setcounter{secnumdepth}{5} %pour la numérotation dans le corps du document
\setcounter{tocdepth}{1} %pour l'apparition dans la table des matières
\titleformat{\section}
{\color{bleudvp}\normalfont\sffamily\Huge\bfseries}
{}{0em}{}
\titlecontents{section}%
  [0em]% <-- à modifier
  {\vspace{0em}}
  {}% <-- à modifier
  {\hspace{-0em}}% <-- à modifier
  {\hfill\color{bleudvp} Page \contentspage}
\usepackage{zref-base,zref-lastpage}
\usepackage{etexcmds}
\usepackage{pdfescape}  
\usepackage{multicol}
\usepackage{fancybox}
\usepackage{tikz}
\usepackage[francais]{babel}
\usepackage[hyphens]{url} 
\usepackage{hyperref}
	\hypersetup{breaklinks=true,linkcolor=bleudvp, urlcolor=bleudvp, colorlinks=true, hypertexnames=false}

%%%%%%%%%%%%%%%%%%%%%couleurs%%%%%%%%%%%%%%%%%%%%%%%%%%
\usepackage{colortbl}
\usepackage{xcolor}
	\definecolor{orangeattention}{RGB}{255,193,71}
	\definecolor{bleudvp}{RGB}{32,30,105} 
	\definecolor{rougedvp}{RGB}{252,4,4} 
	\definecolor{codegreen}{RGB}{91,255,151}
	\definecolor{jauneidee}{RGB}{255,247,147}
	\definecolor{vertperso}{RGB}{204,255,153}
	%\definecolor{macouleurA}{rgb}{0,0.65,0}
	%\definecolor{Numbers}{rgb}{255,128,0}
	\definecolor{gris25}{gray}{0.75}
	\definecolor{gris50}{gray}{0.5}
	\definecolor{blue25}{rgb}{189,183,107}
	%\definecolor{editorGray}{rgb}{0.95, 0.95, 0.95}
	%\definecolor{editorOcher}{rgb}{1, 0.5, 0}
	%\definecolor{editorGreen}{rgb}{0, 0.5, 0}
%\definecolor{darkgray}{rgb}{.4,.4,.4}
\definecolor{purple}{rgb}{0.65, 0.12, 0.82}
%\definecolor{pblue}{rgb}{0.13,0.13,1}
%\definecolor{pgreen}{rgb}{0,0.5,0}
%\definecolor{pred}{rgb}{0.9,0,0}
%\definecolor{pgrey}{rgb}{0.46,0.45,0.48}

\definecolor{darkgreen}{rgb}{0.05, 0.5, 0.06}
\definecolor{DVPlightgray}{rgb}{.9,.9,.9}

\usepackage{listings}
\lstset{%
                inputencoding=utf8,
                    extendedchars=true,
                    literate=%
                    {é}{{\'e}}{1}%
                    {è}{{\`e}}{1}%
                    {à}{{\`a}}{1}%
                    {ç}{{\c{c}}}{1}%
                    {œ}{{\oe}}{1}%
                    {ù}{{\`u}}{1}%
                    {É}{{\'E}}{1}%
                    {È}{{\`E}}{1}%
                    {À}{{\`A}}{1}%
                    {Ç}{{\c{C}}}{1}%
                    {Œ}{{\OE}}{1}%
                    {Ê}{{\^E}}{1}%
                    {ê}{{\^e}}{1}%
                    {î}{{\^i}}{1}%
                    {ô}{{\^o}}{1}%
                    {û}{{\^u}}{1}%
                    {…}{{\ldots}}{1}%
                    {«}{{\og}}{1}%
                    {»}{{\fg}}{1}%
                    {ë}{{\¨{e}}}1
                    {û}{{\^{u}}}1
                    {â}{{\^{a}}}1
                    {Â}{{\^{A}}}1
                    {Î}{{\^{I}}}1
                    {→}{{$\rightarrow{}$}}1
                    {`}{{$\backprime{}$}}1
                    {Θ}{{$\Theta{}$}}1
                    {Σ}{{$\Sigma{}$}}1
                    {Ω}{{$\Omega{}$}}1
                    {Ι}{{I}}1
                    {Χ}{{X}}1
                    {Υ}{{$\Upsilon{}$}}1
                    {Α}{{A}}1
                    {0}{{{\color{rougedvp}0}}}1
                    {├}{{$\vdash$}}1
                    {─}{{$-$}}1
                    {└}{{$\lfloor$}}1
                    {│}{{$|$}}1
    {1}{{{\color{rougedvp}1}}}1
    {2}{{{\color{rougedvp}2}}}1
    {3}{{{\color{rougedvp}3}}}1
    {4}{{{\color{rougedvp}4}}}1
    {5}{{{\color{rougedvp}5}}}1
    {6}{{{\color{rougedvp}6}}}1
    {7}{{{\color{rougedvp}7}}}1
    {8}{{{\color{rougedvp}8}}}1
    {9}{{{\color{rougedvp}9}}}1
            }
    \lstdefinelanguage{dvp}{%
keywords={typeof, new, true, false, catch, function, return, catch, switch, var, if, in, while, do, else, case, break, const},
ndkeywords={glCreateShader, std, VertexShaderStream, getline, FragmentShaderStream, ios, printf,:, glShaderSource, glCompileShader, glGetShaderiv, VertexShaderErrorMessage, glCreateProgram, glAttachShader, glLinkProgram, glGetProgramiv, glGetProgramInfoLog, glGetShaderInfoLog,LoadShaders, fprintf, glDeleteShader, class, export, boolean, throw,  implements, import, this},
%otherkeywords={ =, +, -, \}, \{, *, \&, >, <},
          	keywordstyle=\color{purple}\bfseries,
          	ndkeywordstyle=\color{bleudvp}\bfseries,
         	 identifierstyle=\color{black},
         	 sensitive=false,
         	 comment=[l]{//},
          	morecomment=[s]{/*}{*/},          
          	commentstyle=\color{darkgreen}\ttfamily,
          	stringstyle=\color{blue}\ttfamily,
          	xleftmargin={0.75cm},
    		%alsodigit={:},
         	morestring=[b]',
          	%morestring=[b]>,
          	%morestring=[b]<,
          	morestring=[b]"
        }

        \lstset{%
           language=dvp,
           emph={char, int},
           emphstyle=\color{blue!50}\bfseries,
           emph={[2]Null},
           emphstyle=[2]\color{codegreen},
           backgroundcolor=\color{DVPlightgray},
           extendedchars=true,
           basicstyle=\footnotesize\ttfamily,
           showstringspaces=false,
           showspaces=false,
           numbers=left,
           numberstyle=\footnotesize,
           numbersep=9pt,
           firstnumber=1,
           numberfirstline=true,
           stepnumber=1,
           tabsize=2,
           breaklines=true,
           showtabs=false,
           captionpos=b             
        }

\makeatother


\renewcommand{\thesubsection}{\arabic{subsection}}
\makeatletter
\newcounter{subsubsubsection}[subsubsection]
\renewcommand\thesubsubsubsection{\thesubsubsection .\@alph\c@subsubsubsection}
\newcommand\subsubsubsection{\@startsection{subsubsubsection}{4}{\z@}%
                                     {-3.25ex\@plus -1ex \@minus -.2ex}%
                                     {1.5ex \@plus .2ex}%
                                     {\normalfont\normalsize\bfseries}}
\newcommand*\l@subsubsubsection{\@dottedtocline{3}{10.0em}{4.1em}}
\newcommand*{\subsubsubsectionmark}[1]{}
\makeatother

\makeatletter
\newcounter{subsubsubsubsection}[subsubsubsection]
\renewcommand\thesubsubsubsubsection{\thesubsubsubsection .\@alph\c@subsubsubsubsection}
\def\toclevel@subsubsubsection{4}
\newcommand\subsubsubsubsection{\@startsection{subsubsubsubsection}{4}{\z@}%
                                     {-3.25ex\@plus -1ex \@minus -.2ex}%
                                     {1.5ex \@plus .2ex}%
                                     {\normalfont\normalsize\bfseries}}
\newcommand*\l@subsubsubsubsection{\@dottedtocline{3}{10.0em}{4.1em}}
\newcommand*{\subsubsubsubsectionmark}[1]{}
\makeatother


\makeatletter
\newcounter{subsubsubsubsubsection}[subsubsubsubsection]
\renewcommand\thesubsubsubsubsubsection{\thesubsubsubsubsection .\@alph\c@subsubsubsubsubsection}
\def\toclevel@subsubsubsubsection{4}
\newcommand\subsubsubsubsubsection{\@startsection{subsubsubsubsubsection}{4}{\z@}%
                                     {-3.25ex\@plus -1ex \@minus -.2ex}%
                                     {1.5ex \@plus .2ex}%
                                     {\normalfont\normalsize\bfseries}}
\newcommand*\l@subsubsubsubsubsection{\@dottedtocline{3}{10.0em}{4.1em}}
\newcommand*{\subsubsubsubsubsectionmark}[1]{}
\makeatother

% Gestion des notes de bas de pages au sein de l'environnement minipage
%\usepackage{footnote}
%\makesavenoteenv{minipage}